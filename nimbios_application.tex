\documentclass{article}

\begin{document}
%  Title (descriptive title of planned research, 80 characters max)
% Name and contact information

\section{Project Summary} 
%(200 words max) - appropriate for public distribution on the NIMBioS web site
% 4 Lines on Gilchrist's summary
% 1 line on our application


Many problems in computational biology require efficient sampling
methods in order to make inferences from large-scale, high-dimensional
datasets.  While Markov Chain Monte Carlo (MCMC) methods and
ubiquitous computing power have allowed researchers to explore
previously intractable distributions, there remain entire classes of
problems for which existing MCMC algorithms prove inadequate.  One
such class includes Bayesian posterior sampling problems which require
evaluation of a likelihood function that can only be computed exactly
up to a normalizing constant.  A second, nested application of MCMC is
required to estimate the normalization constant, leading researchers
to refer to such distributions as ``doubly intractable''\cite{murray06}.

In this project we aim to develop a software library and validation
suite to facilitate sampling from doubly intractable distributions.
We intend for our computational tools to be of broad utility to
researchers in the biological and statistical sciences by providing a
unified interface to existing algorithms for sampling from doubly
intractable distributions as well as integrated support for cluster
computing resources.  Applications to the authors' own work in the
evolution of coding sequences and gene regulatory networks are immediate.


\section{Introduction and Goals}
%- what important problem at the math/biology interface is being addressed?
% Generic goal: software library for doubly intractable MCMC
% Doubly intractable MCMC problems arise in many biological applications
Often in computational biology one wishes to sample from a posterior
distribution $P(\theta|y)$ of models belonging to some class $\Theta$
given data $y$.  In Metropolis-Hastings sampling, one need only be
able to compute the prior probability $P(\theta)$ and the likelihood
$P(y|\theta)$ in order to construct a sampling algorithm for
$P(\theta|y)$.  In many contexts, however, the likelihood
$P(y|\theta)$ is itself difficult to compute---often it is only known
up to a normalization constant $Z(\theta)$ which usually can only be
estimated through a second round of Metropolis-Hastings sampling.  The
problem of sampling from ``doubly intractable'' distributions
therefore poses a difficult challenge to the computational biologists
who encounter them in such diverse areas as molecular evolution,
phylogenetics and the analysis of gene regulatory networks.

Our primary aim is to develop a software library which embodies the
state of the art in sampling algorithms for doubly intractable
distributions.  In particular, we intend to implement the Auxiliary
Variable Method \cite{moller06}, Single Variable Exchange Algorithm
\cite{murray06}, and the adaptive Wang-Landau-type MCMC algorithm
described in \cite{atchade08}.  Due to the embarrassingly parallel
nature of the sampling problem, we also intend for our library to
efficiently leverage the capabilities of multi-core desktops and
cluster computing environments without additional overhead for the end
user.  Despite the centrality of doubly intractable distributions to
many areas of biological research, resources for sampling from them
are generally lacking, and we consider this state of affairs a
significant impediment to research in computational biology.  

Once development is complete and the software is released to the
community, we envision several immediate applications.  For example,
the applicant's current research focuses on the evolution of
transcriptional regulation.  To that end, we have developed a detailed
\textit{in silico} biophysical model of transcription factor / binding
site coevolution which allows us to study the properties of
transcriptional systems which evolve to solve various regulatory
tasks.  Often, however, we wish to pose the inverse problem: given a
partially observed regulatory system, \textit{i.e.} a collection of
experimentally-verified binding sites, what is the posterior
distribution over the space of hypotheses about the original
regulatory requirements?  The resulting distribution is in fact doubly
intractable, as the fitness $P(y|\theta)$ of a given regulatory system
$y$ given an evolutionary environment $\theta$ depends on the fitness of
all other possible systems under that environment.  

Another application arises from the observation that the collection of
binding sites is only a subset of the full regulatory system, which at
the very least includes features such as an energy model of the
DNA-binding domain of the transcription factor, its copy number, and
so on.  This leads to a missing data problem in which one wishes to
sample from the posterior $P(\theta|y)$ when one has only the data
$y'$ where $y'\subset y$.  Such problems can be shown to require the
evaluation of integrals of the form $\int_{y|y'\subset
  y}P(y|\theta)dy$.  Such distributions are therefore \textit{triply}
intractable, and require additional research in order to obtain
efficient sampling algorithms.


% For example, using this library w/ ESTReMo allows us to address...
\section{Proposed Activities and Rationale for NIMBioS Support}
We feel that NIMBioS provides an unparalleled opportunity to
collaborate with scientists working at the interface of biological and
mathematical research.  Although the proposed project can be
considered as a purely statistical problem, it arose directly from
practical issues in modeling evolutionary biology.  In fact, it arose
independently in our work and in the work of Michael Gilchrist, with
whom we are very interested in collaborating at NIMBioS.  Therefore,
we feel that the methods we intend to implement will be of very
general use, and consider the environment at NIMBioS to uniquely
suited for soliciting feedback from other mathematical biologists in
order to produce the most broadly useful tools for the research
community.

We respectfully request support for travel and lodging, as well as
access to NIMBioS's high performance computing facilities for the
purpose of developing and testing the parallelization of our sampling
library.

% - Why can this activity be most effectively conducted at NIMBioS? Also
% specify what support you are requesting from NIMBioS (travel, lodging,
% etc.).

% NIMBioS is a unique buzzword buzzword buzzword... Jiang, Russ, Mike, &c. ("resource")

\section{Anticipated IT Needs} 
% - Briefly describe any needs for IT support that
% are important to the success of the proposed project, including
% support for coding, high performance computing, and development and
% maintenance of public databases.

We will develop all of our own software, but it would be very useful
to have access to NIMBioS's HPC facilities for development and testing
of parallel algorithms.  While we possess prior experience with
cluster computing environments, we acknowledge the differences which
exist between systems and might perhaps benefit from a short
introduction to NIMBioS's facilities by a member of the scientific
computing staff.

\section{Proposed Collaborators} 
% (if any) and 
We are primarily interested in collaborating with Michael Gilchrist
and his colleagues Russell Zaretzki and Jian Huang at University of
Tennessee, Knoxville.  We would also, however, be very excited to meet
broadly with the NIMBioS community in order to identify applications
or extensions for our software in other projects.

\section{Timetable}

We envision a short-term visit on the order of one week, beginning on
or about Wednesday November 13, 2013, lasting until Friday, November
22 2013.  During that period we intend to accomplish the following:
\begin{enumerate}
\item Preliminaries
  \begin{enumerate}
  \item Introductory meetings with principal collaborators
  \item Presentation of literature review
  \item Meetings with other members of NIMBioS community (throughout)
  \item Formal project definition
  \end{enumerate}
  % Project definition,
  % language negotiations, parallelization options, benchmark/validation
  % suite
\item Development
  \begin{enumerate}
  \item Implementation of algorithms
  \item Parallelization and meetings with scientific computing staff
  \item Generalization of statistical framework to treat missing data problems
  \end{enumerate}
\item Testing
  \begin{enumerate}
  \item Validation and benchmarking
  \item Application to Dr. Gilchrist's work in the evolution of coding sequences
  \item Application to ESTReMo
  \item Other applications to NIMBioS community.
  \end{enumerate}
\item Conclusion
  \begin{enumerate}
  \item Release of open-source library
  \item Definition of further project aims to be concluded remotely.
  \end{enumerate}
\end{enumerate}
% (including start
% date). Local collaborators must send a short letter to NIMBioS
% confirming they are interested and available. The letter can be sent
% with the application or sent separately to Chris Welsh at
% cwelsh@nimbios.org.  

\section{Anticipated Results}
\begin{itemize}
  \item Software library
  \item Performance study on benchmark/validation suite
  \item Generalization to missing data problems
  \item Gilchrist's application
  \item Estremo application
  \item pubs?
\end{itemize}

Short CV of the Applicant (2 pages) 

\bibliography{../bib/bibliography}{}
\bibliographystyle{abbrv} \newpage
\end{document}